The {\tt Iblock2} input deck in the example subdirectory describes a
square mesh with {\tt n} elements on a side, where the parameter {\tt n}
is not defined in the input deck.  We start a FEAP simulation with
{\tt n = 10}, get the tangent stiffness and residual into MATLAB,
solve the linear system and write the results back to FEAP, and then
use FEAP's X11 graphics to show the displaced shape.  Once the user
has finished admiring our deformed block, he can press a key (at which
point the FEAP simulation will exit and the graphics will disappear).

\begin{verbatim}
param.n       = 10;  % Parameter to the FEAP input deck
param.verbose = 1;   % See everything that FEAP sends

p  = feapstart('Iblock2', param);  % Start FEAP simulation

K  = feaptang(p);    % Form the tangent matrix
R  = feapresid(p);   % Form the residual force vector
du = K\R;            % Compute a Newton update
feapsetu(p, du);     % Set the displacement vector

% Plot the results
feapcmd(p, 'plot', 'defo', 'mesh', 'boun', 'load', 'end');

% Quit
disp('Press any key to exit');
pause;
feapquit(p);

\end{verbatim}
